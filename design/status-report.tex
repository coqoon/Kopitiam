\documentclass[11pt]{exam}

\usepackage{listings}
\lstset{basicstyle=\footnotesize, showstringspaces=false,columns=fullflexible, basicstyle=\footnotesize\ttfamily, frame=tb }
\usepackage{pdfsync}
\usepackage{subfigure}
\usepackage{amsmath}
\usepackage{amsfonts}
\usepackage[pdftex]{graphicx}
\usepackage{enumitem}
\usepackage{fullpage}

\setlength{\parindent}{0in} % Removing indentation of new paragraphs

\title{Status Report}
\author{Mads Hartmann Jensen}

\begin{document}

\maketitle{}

\paragraph{} This report contains the details of my work for Hannes on his PhD project Kopitiam during my employment. The purpose is not to explain the theory behind the work but rather give an overview of what I've worked on and the state of completion as I leave the project. the My work includes build system setup and different kinds of static analysis of Java programs. \newline \newline

\tableofcontents

\newpage

\section{Build System}

I switched the build system of the project to use the popular Scala build tool SBT (Simple Build Tool) and set up support for automated tests and automatic test coverage reports.

\section{Live variable analysis}

I implemented a live variable analysis of Java programs. The goal of the analysis is to remove any dead variables introduced by our conversion from Java to Simple Java rather then attempt to improve the original Java programs written by the user of Kopitiam.

\subsection{Details}

The implementation can be found in the file \texttt{src/main/scala/dk/itu/sdg/analysis/Optimizer.scala} and the tests can be found in \texttt{src/test/scala/dk/itu/sdg/analysis/OptimizerTest.scala}. \newline

TODO: Write a little about the data-structures and high-level algorithm.

\subsection{State of Completion}

TODO: I'll write this when the two last tests pass.

\newpage

\section{Purity analysis}

I've implemented a Purity analysis of Java programs based on the paper A Combined Pointer and Purity Analysis for Java Program by Alexandru Salcianu and Martin Rinard. I also presented the algorithm to the Tools and Methods for Scalable Software Verification (TOMESO) group at ITU.

\subsection{Details}

The implementation can be found in the file \texttt{src/main/scala/dk/itu/sdg/analysis/Purity.scala} and the tests can be found in \texttt{src/test/scala/dk/itu/sdg/analysis/PurityTest.scala} and \texttt{src/test/scala/dk/itu/sdg/analysis/RegExpGeneratorTest.scala}. \newline

I will attempt to give a high-level overview of the implementation covering the entry points and which pieces of code do what and. For a deeper understanding of the implementation I suggest reading the code and the paper side-by-side. \newline

\texttt{analysis(className: String, invokable: SJInvokable): Result} is the main entry point to the analysis. \texttt{className} is the name of the class that contains the method (this is only needed because \texttt{SJInvokable} doesn't have a reference to the class in which it is defined) and \texttt{invokable} is the method (or constructor) that you want to analyze. \texttt{Result} is simply a \texttt{case class} that contains the points-to graph of the method and the set of modified abstract fields. This method takes care of the fixed-point iteration when analyzing each of the strongly connected components of the call graph. \newline

The method \texttt{bulkTransfer} takes care of analyzing a list of consecutive \texttt{SJStatement}s. It simply traverses the statements from top to bottom and uses the transfer functions to transform the points-to graph on every statement on the way and (possibly) adding AbstractField to the writes set. The object \texttt{TransferFunctions} contains methods for each of the transfer functions as defined in the paper. The \texttt{case class TFState} is used to model the state as each of the statements are being processed. \newline

\texttt{mapping} creates the appropriate node-mapping used when two points-to graphs needs to be merged and \texttt{combine} takes case of the actual merging of the two graphs. \newline

No mutable state is used in the implementation.

\subsection{State of Completion}

Largely the implementation works. For each of the examples in the paper the implementation is successful in detecting if a method is pure or not. Currently the test \texttt{PurityAnalysisExample.flipAll} fails but this is only because the modified path strings aren't being compressed, see \ref{subsub:mp}

\subsubsection{Points-to graph simplification}

The paper explain how to simplify the points-to graph after mering two points-to graphs after a method invocation. The current implementation doesn't do any of these simplifications. I have prepared the method \texttt{simplify} but it's currently the identity function.

\subsubsection{Static Fields}

There are a couple of aspects of Java that the article covers but the current implementation doesn't. These are:

\begin{itemize}
  \setlength{\itemsep}{1pt}
  \setlength{\parskip}{0pt}
  \item Arrays
  \item Unanalyzable methods, i.e. methods where we don't have access to the source.
  \item Static Fields
\end{itemize}

As I understand the two first points can be ignore in Kopitiam but the third one might be of interest. I have implemented the transfer functions related to static fields but due to a problem\footnote{See the mail to Hannes in the appendix describing this error in some more detail.} somewhere when the SimpleJava AST is being generated I haven't been able to complete the implementation. I have, however, added TODO comments in the code that describe what would need to be changed.

\subsubsection{Modified Paths}
\label{subsub:mp}

With the current implementation it's possible to generate string that show the paths of the objects that are being modified (if any). I still need to add some compression to the strings such that

\begin{itemize}
  \setlength{\itemsep}{1pt}
  \setlength{\parskip}{0pt}
  \item foo.bar.x
  \item foo.bar.y
  \item foo.bar.next.x
  \item foo.bar.next.y
  \item foo.bar.next.next.x
  \item foo.bar.next.next.y
\end{itemize}

turns into: \texttt{foo.bar.next*.(x|y)}

\subsection{Things not mentioned in the paper}

There is one thing that wasn't completely clear from the paper and one thing that was neglected

\begin{itemize}
  \item When a statement is of the form \texttt{v = new C} then in addition to invoking the transfer function as described in the paper it should also be considered a method invocation.
  \item The article doesn't mention entirely how to deal with while loops so I had to improvise: let \texttt{G} be the graph before the loop iteration and let \texttt{G'} be the graph produced by the loop iteration. Consider the \texttt{Edge(n1,f,n2)} from \texttt{G'} that doesn't exist in \texttt{G}, i.e. an edge produced by the iteration. If there exists an \texttt{Edge(n1,f,n3)} in \texttt{G} then map the node \texttt{n2} to \texttt{n3}.
\end{itemize}

\newpage

\section{Possible Code Improvements For When There's More Than 24h In A Day}

This is a list of various possible improvements to the parts of the code base of Kopitiam that I've worked on.

\subsection{Lenses}

In the purity analysis I use case classes to model the state of the algorithm. In an effort to stay sane these case classes are all immutable so I use the \texttt{copy} method when I need to make changes to the data structure. However, this gets quites messy when I want to change something deep in the data structure as shown below.

\begin{figure}[h!]
  \begin{lstlisting}
  state.copy(
    result = state.result.copy(
      pointsToGraph = ptGraph(state).copy(
        stateOflocalVars = localVars(state).updated(v1, nodes)
      )
    )
  )
  \end{lstlisting}
\end{figure}

This answer\footnote{http://stackoverflow.com/questions/3900307/cleaner-way-to-update-nested-structures\#answer-5597750} on Stack Overflow contains information about lenses with references to a CS paper and blog posts with examples in Scala. You could hand-code the lenses for the appropriate case classes or if you're feeling brave use this compiler plugin\footnote{https://github.com/gseitz/Lensed} that generates them for you. \newline

With lenses you could change to code to something like this:

\begin{figure}[h!]
  \begin{lstlisting}
    state.result.pointsToGraph.stateOfLocalVars.mod(state, _.updated(v1, nodes ))
  \end{lstlisting}
\end{figure}

\subsection{Type-safe AST transformations}

Currently when we're transforming the AST during static analysis we're using casting to make it compile. We could instead use the same approach as they do in the Scala compiler\footnote{https://github.com/scala/scala/blob/master/src/library/scala/reflect/api/Trees.scala\#L1035} where we simply define a Transformer abstract class with methods for each of the AST nodes we want to be able to rewrite. For a specific rewrite of the tree we then create a new instance of Transformer and override the tranformXXX method of the node we want to work with. An example of this is shown in figure \ref{transform_ast_scalac}.

\section{Appendix}

\begin{figure}[b]
\begin{lstlisting}
Hi Hannes,

FYI, the following example makes the AST transformation to SimpleJavaAST crash:

class Person {

 static String defaultName = "Mr. Default";

 String name;

 public Person(String s) {
   this.name = s;
 }
}

class AssignToStaticField {
 public void setDefaultName(String defaultName) {
   Person.defaultName = defaultName;
 }
}

I get the exception: java.util.NoSuchElementException: key not found: Person

And it's because of the line: Person.defaultName = defaultName;

You can invoke this by running:

test-only dk.itu.sdg.analysis.PurityTestsByMads

or

run src/test/resources/static_analysis/source/AssignToStaticField.java

and hit 1.
\end{lstlisting}
  \caption{Mail to Hannes describing error in the AST transformation to SimpleJava AST}
  \label{mail_to_hannes}
\end{figure}

\begin{figure}[b]
\begin{lstlisting}
object Test {

  def main(args: Array[String]): Unit = {
    abstract class Transformer {

      def transform(defi: Definition): Class = defi match {
        case Class(name, body) => Class(transformName(name), body map {transformStatement(_)})
      }

      def transformName(name: String): String = name

      def transformValue(value: Int): Int = value

      def transformStatement(stm: Statement): Statement = stm match {
        case VariableWrite(id, value) => VariableWrite(transformName(id), transformExpression(value))
        case Return(expr) => Return(transformExpression(expr))
      }

      def transformExpression(exp: Expression): Expression = exp match {
        case VariableAccess(id) => VariableAccess(transformName(id))
        case Value(value) => Value(transformValue(value))
      }
    }

    // AST definition
    sealed trait Transformable
    sealed trait Definition extends Transformable
    case class Class(name: String, body: List[Statement]) extends Definition

    sealed trait Statement extends Transformable
    case class VariableWrite(id: String, value: Expression) extends Statement
    case class Return(expr: Expression) extends Statement

    sealed trait Expression extends Transformable
    case class VariableAccess(id: String) extends Expression
    case class Value(value: Int) extends Expression

    // Build a simple test AST
    val ast = Class("Thingy", List(
      VariableWrite("foo", Value(42)),
      Return(VariableAccess("foo"))
    ))

    // Do some transformations.
    println((new Transformer {
      override def transformName(name: String): String = name.reverse
      override def transformValue(value: Int): Int = value + 10
    }) transform ast)
  }
}
\end{lstlisting}
  \caption{Possible solution to type-safe tree re-write}
  \label{transform_ast_scalac}
\end{figure}

\end{document}